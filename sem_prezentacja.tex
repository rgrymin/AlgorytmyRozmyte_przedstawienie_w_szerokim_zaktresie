\input{preamble}


\begin{document}
% \usebackgroundtemplate{\includegraphics[width=\paperwidth]{img/bg.jpg}}
\frame{\titlepage}

% Przebieg prezentacji
\begin{frame}
  \frametitle{Przebieg prezentacji}
  \tableofcontents
\end{frame}

% The following causes the table of contents to be shown 
% at the beginning of every subsection. Delete this, if you do not want it.
\AtBeginSubsection[]{
  \frame<beamer>{
    \frametitle{Przebieg prezentacji}
    \tableofcontents[currentsection,currentsubsection]
  }
}


\section{Jak wygląda pętla sprzężenia zwrotnego 
    sterująca odwróconym wahadłem}
    
    \begin{frame}{Układ Automatycznej Regulacji (UAR)}
    	\begin{figure}[!htp]
    		\centering
    		\includegraphics[width=\textwidth]{img/uar}
    		\caption{UAR}
    	\end{figure}
    \end{frame}
    

    
\section{Rodzaje stosowanych sterowników}
	\begin{frame}{PID}
    	\begin{figure}[!htp]
    		\centering
    		\includegraphics[width=\textwidth]{img/pid_real}
    		\caption{Rzeczywisty regulator PID}
    	\end{figure}
	\end{frame}
	
	\begin{frame}{Kaskada}
    	\begin{figure}[!htp]
    		\centering
    		\includegraphics[width=\textwidth]{img/kaskada}
    		\caption{Układ kaskadowy}
    	\end{figure}
	\end{frame}
	
	\begin{frame}[t]{Sterownik rozmyty}
		\only<1>
		{
    		\begin{figure}[!htp]
    			\centering
    			\includegraphics[width=0.48\textwidth]{img/fuzzy}
    			\caption{Układ kaskadowy}
    		\end{figure}\pause
    	}
    	\only<2->
    	{
    		\begin{itemize}
    			\item Podstawowe pojęcia. 
    			  \pause \pause
    			\item Zasada działania sterownika rozmytego. 
    			  \pause
    			\item Przykłady kształtów funkcji przynależności.
    		\end{itemize}
    	}
	\end{frame}
  
\section{Możliwe rozwiązania rozmyte}
	\begin{frame}[t]{Czysty sterownik rozmyty}
		\begin{itemize}
			\item nie wykorzystujemy modelu z uwagi na
				nieliniowość i trudne analitycznie algorytmy
				sterowania
				  \pause
				
			\item można poprawnie wysterować odwrócone wahadło
				przy użyciu bazy reguł 
				\pause
				
			\item łatwa implementacja algorytmu nieliniowej regulacji
			    mimo trudnego modelu obiektu
		\end{itemize}
	\end{frame}
	
	\begin{frame}[t]{Sterownik w kaskadzie z innymi sterownikami}
		\begin{itemize}
			\item używany w przypadku, gdy posiadany jest model,
				ale właściwości sterownika wymagają korekcji
				\pause
				
			\item pomaga w łatwym kształtowaniu charakterystyki częstotliwościowej
		\end{itemize}
	\end{frame}
  


\section{}
\begin{frame}
  \begin{center}
    \huge
    Dziękujemy za uwagę!
  \end{center}
 

\end{frame}

\end{document}
